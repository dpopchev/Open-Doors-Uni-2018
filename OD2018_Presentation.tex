\documentclass[10pt,a4paper]{beamer}
\usepackage[utf8]{inputenc}
\usepackage[english,bulgarian]{babel}
\usepackage{amsmath, amsfonts, amssymb, amsthm, bigints, textcomp, caption, subcaption, float, graphicx, tikz, xcolor}
\usetikzlibrary{arrows,shapes,positioning}
\newcommand\ytl[2]{
    \parbox[b]{8em}{\hfill{\color{cyan}\bfseries\sffamily #1}~$\cdots\cdots$~}\makebox[0pt][c]{$\bullet$}\vrule\quad \parbox[c]{4.5cm}{\vspace{7pt}\color{red!40!black!80}\raggedright\sffamily #2.\\[7pt]}\\[-3pt]
}

\usetheme{Madrid}

\title{ Скаларно тензорни теории - новата истина за гравитацията }
\author{ Димитър Попчев }
\institute{ Софийски Университет „Св. Климент Охридски“ }
\titlegraphic{ \includegraphics[scale=0.25]{images/GR_crack_intro.png} }

\newcommand*{\yellowemph}[1]{%
	\tikz[baseline]\node[rectangle, fill=yellow, rounded corners, inner sep=0.3mm,anchor=base]{#1};%
}

\begin{document}
	
	\begin{frame}
		\titlepage
	\end{frame}
	
	\section{ История }

		\begin{frame}{}
			\tableofcontents[currentsection]
		\end{frame}
		
		\begin{frame}{ Исторически бележки }
            \begin{table}
                \centering
                \begin{minipage}[t]{.7\linewidth}
                    \color{gray}
                    \rule{\linewidth}{1pt}
                    \ytl{$\sim$330 BCE}{Аристотелова механика}
                    \ytl{1589}{Галилеева механика}
                    \ytl{1687}{Нютонова механика}
                    \ytl{1859}{Аномалия на перихелия на Меркурий}
                    \ytl{1916}{Обща теория на относителността}
                    \ytl{1961}{Теория на Бранс-Дике}
                    \ytl{1979}{Скаларно-тензорни теории}
                    \bigskip
                    \rule{\linewidth}{1pt}%
                \end{minipage}%
            \end{table}
		\end{frame}
        
        \begin{frame}{ Аристотелова механика }
			\begin{columns}
				\begin{column}{0.6\textwidth}
					\includegraphics[width=\textwidth]{images/aristotle_four_elements_motion.jpg}
				\end{column}
				\begin{column}{0.4\textwidth}
					\begin{itemize}
						\item Четири основни елемента
                        \item Естествено движение \begin{itemize}
                            \item $ v \sim m $
                            \item $ v \sim 1/b $
                        \end{itemize}
                        \item a.animate\_all\_aristotel()
					\end{itemize}
				\end{column}
			\end{columns}
        \end{frame}
    
        \begin{frame}{ Аристотелова механика }
			\begin{columns}
				\begin{column}{0.6\textwidth}
					\includegraphics[width=\textwidth]{images/aristotle_four_elements_motion.jpg}
				\end{column}
				\begin{column}{0.4\textwidth}
					\begin{itemize}
						\item Четири основни елемента
                        \item Естествено движение 
                        \item Неестествено движение
					\end{itemize}
				\end{column}
			\end{columns}
        \end{frame}
		
        \begin{frame}{ Аристотелова механика }
            \begin{center}
                
                \includegraphics[scale=0.3]{images/aristotel_universe.png}
                
            \end{center}
        \end{frame}

        \begin{frame}{ Наследство на Аристотел }
            \begin{columns}
                \begin{column}{0.5\textwidth}
                    \includegraphics[width=0.9\textwidth]{images/aristotle_portrait.jpg}
                \end{column}
                \begin{column}{0.5\textwidth}
                    \begin{itemize}
                        \item Научно познание
                        \item Модел на Вселената
                    \end{itemize}
                \end{column}
            \end{columns}
        \end{frame}
    
        \begin{frame}{ Наследство на Аристотел }
            \begin{columns}
                \begin{column}{0.5\textwidth}
                    \includegraphics[width=\textwidth]{images/aristotel_vs_galiele.png}
                \end{column}
                \begin{column}{0.5\textwidth}
                    \begin{itemize}
                        \item Научно познание
                        \item Модел на Вселената \\
                        \noindent\rule{5cm}{0.5pt}
                        \item ... до Галилей
                    \end{itemize}
                \end{column}
            \end{columns}
        \end{frame}
    
        \begin{frame}{ Галилеева механика }
            \begin{columns}
                \begin{column}{0.5\textwidth}
                    \includegraphics[width=\textwidth]{images/galilei_ramps.jpg}
                \end{column}
                \begin{column}{0.5\textwidth}
                    \begin{itemize}
                        \item Експерименти с рампи
                    \end{itemize}
                \end{column}
            \end{columns}
        \end{frame}
    
        \begin{frame}{ Галилеева механика }
            \begin{columns}
                \begin{column}{0.5\textwidth}
                    \includegraphics[width=0.8\textwidth]{images/galilei_aristotle_piza.jpg}
                \end{column}
                \begin{column}{0.5\textwidth}
                    \begin{itemize}
                        \item Експерименти с рампи \begin{itemize}
                            \item $ \Delta v \sim g $
                            \item $ s \sim t^2 $
                        \end{itemize}
                        \item a.animate\_all\_galilee\_vs\_A()
                    \end{itemize}
                \end{column}
            \end{columns}
        \end{frame}
    
        \begin{frame}{ Галилеева механика }
            \begin{columns}
                \begin{column}{0.5\textwidth}
                    \includegraphics[width=\textwidth]{images/galieli_inqusition.jpg}
                \end{column}
                \begin{column}{0.5\textwidth}
                    \begin{itemize}
                        \item Експерименти с рампи
                        \item Галилей срещу църквата
                    \end{itemize}
                \end{column}
            \end{columns}
        \end{frame}
        
        \begin{frame}{ Галилеева механика }
            \begin{columns}
                \begin{column}{0.5\textwidth}
                    \includegraphics[width=\textwidth]{images/galilei_venus_ptolomey.png}
                \end{column}
            \end{columns}
        \end{frame}
        
        \begin{frame}{ Галилеева механика }
            \begin{columns}
                \begin{column}{0.3\textwidth}
                    \includegraphics[width=\textwidth]{images/galilei_venus_ptolomey.png}
                \end{column}
                \begin{column}{0.7\textwidth}
                    \includegraphics[width=\textwidth]{images/galilei_venus_copernic.png}
                \end{column}
            \end{columns}
        \end{frame}
        
        \begin{frame}{ Галилеева механика }
            \begin{columns}
                \begin{column}{0.5\textwidth}
                    \includegraphics[width=\textwidth]{images/aristotel_universe.png}
                \end{column}
            \end{columns}
        \end{frame}
        
        \begin{frame}{ Галилеева механика }
            \begin{columns}
                \begin{column}{0.2\textwidth}
                    \includegraphics[width=\textwidth]{images/aristotel_universe.png}
                \end{column}
                \begin{column}{0.6\textwidth}
                    \includegraphics[width=\textwidth]{images/galilei_copernic_model.png}
                \end{column}
            \end{columns}
        \end{frame}
        
        \begin{frame}{ Нютонова механика }
            \begin{columns}
                \begin{column}{0.5\textwidth}
                    \includegraphics[width=\textwidth]{images/newton_apple.jpg}
                \end{column}
            \end{columns}
        \end{frame}
        
        \begin{frame}{ Нютонова механика }
            \begin{columns}
                \begin{column}{0.5\textwidth}
                    \includegraphics[width=0.9\textwidth]{images/newton_apple.jpg}
                \end{column}
                \begin{column}{0.5\textwidth}
                    \begin{itemize}
                        \item Трите принципа на Нютон \begin{itemize}
                            \item Принцип за инерцията
                            \item Принцип за резултатна сила
                            \item Принцип за действие - противодействие
                        \end{itemize}
                    \end{itemize}
                \end{column}
            \end{columns}
        \end{frame}
        
        \begin{frame}{ Нютонова механика }
            \begin{columns}
                \begin{column}{0.5\textwidth}
                    \includegraphics[width=0.9\textwidth]{images/newton_apple.jpg}
                \end{column}
                \begin{column}{0.5\textwidth}
                    \begin{itemize}
                        \item Трите принципа на Нютон
                        \item $ F^{\prime} = W - F_{drag} $
                        \item a.animate\_all\_newton\_vs\_G\_A()
                    \end{itemize}
                \end{column}
            \end{columns}
        \end{frame}
        
        \begin{frame}{ Нютонова механика }
            \begin{columns}
                \begin{column}{0.5\textwidth}
                    \includegraphics[width=0.9\textwidth]{images/newton_apple.jpg}
                \end{column}
                \begin{column}{0.5\textwidth}
                    \begin{itemize}
                        \item Трите принципа на Нютон
                        \item $ F^{\prime} = W - F_{drag} $
                        \item Закон за Всемирно привличане \begin{itemize}
                            \item $ F = \dfrac{GMm}{r^2} $
                            \item $ g \approx \dfrac{GM}{R^2} = 9.81 \mbox{m/s}^2 $
                        \end{itemize}
                    \end{itemize}
                \end{column}
            \end{columns}
        \end{frame}
        
        \begin{frame}{ Нютонова механика }
            \begin{columns}
                \begin{column}{0.5\textwidth}
                    \includegraphics[width=0.9\textwidth]{images/newton_neptune_discovery.png}
                \end{column}
                \begin{column}{0.5\textwidth}
                    \begin{itemize}
                        \item Трите принципа на Нютон
                        \item $ F^{\prime} = W - F_{drag} $
                        \item Закон за Всемирно привличане
                        \item Откриване на Нептун, 1846!
                    \end{itemize}
                \end{column}
            \end{columns}
        \end{frame}
    
        \begin{frame}{ Нютонова механика }
            \begin{columns}
                \begin{column}{0.5\textwidth}
                    \includegraphics[width=\textwidth]{images/newton_mercury_perih.png}
                \end{column}
                \begin{column}{0.5\textwidth}
                    \begin{itemize}
                        \item Предсказани \begin{itemize}
                            \item $ \sim 532 $ arcsec/century
                        \end{itemize}
                        \item a.animate\_all\_newton\_orbit()
                    \end{itemize}
                \end{column}
            \end{columns}
        \end{frame}
    
        \begin{frame}{ Нютонова механика }
            \begin{columns}
                \begin{column}{0.5\textwidth}
                    \includegraphics[width=\textwidth]{images/newton_mercury_perihelion.jpg}
                \end{column}
                \begin{column}{0.5\textwidth}
                    \begin{itemize}
                        \item Предсказани \begin{itemize}
                            \item $ \sim 532 $ arcsec/century
                        \end{itemize}
                        \item Наблюдавани \begin{itemize}
                            \item $ \sim 574 $ arcsec/century
                        \end{itemize}
                        \noindent\rule{5cm}{0.5pt}
                        \item Разлика от $ \sim 42 $ arcsec/century
                    \end{itemize}
                \end{column}
            \end{columns}
        \end{frame}
    
        \begin{frame}{ Обща теория на относителността }
            \begin{columns}
                \begin{column}{0.8\textwidth}
                    \includegraphics[width=\textwidth]{images/gr_spacetime.jpeg}
                \end{column}
            \end{columns}
        \end{frame}
    
        \begin{frame}{ Обща теория на относителността }
            \begin{columns}
                \begin{column}{0.5\textwidth}
                    \includegraphics[width=\textwidth]{images/einstein_gr.png}
                \end{column}
            \end{columns}
        \end{frame}
        
    \section{ Настояще }

        \begin{frame}{}
            \tableofcontents[currentsection]
        \end{frame}
        
        \begin{frame}{ Парадокс на Олберт }
            \begin{columns}
                \begin{column}{0.5\textwidth}
                    \includegraphics[width=\textwidth]{images/olbers_paradox.png}
                \end{column}
                \begin{column}{0.5\textwidth}
                    \begin{itemize}
                        \item Въпрос от $ \sim 1826 $
                        \item Безкрайна Вселена?
                    \end{itemize}
                \end{column}
            \end{columns}
        \end{frame}
        
        \begin{frame}{ Парадокс на Олберт }
            \begin{columns}
                \begin{column}{1\textwidth}
                    \includegraphics[width=\textwidth]{images/gr_olbert_resolution.jpg}
                \end{column}
            \end{columns}
        \end{frame}
        
        \begin{frame}{ Парадокс на Олберт }
            \begin{columns}
                \begin{column}{0.5\textwidth}
                    \includegraphics[width=\textwidth]{images/big_bang_theory.jpeg}
                \end{column}
                \begin{column}{0.5\textwidth}
                    \begin{itemize}
                        \item Теория за Големия Взрив \begin{itemize}
                            \item Разширяващата се вселена
                        \end{itemize}
                    \end{itemize}
                \end{column}
            \end{columns}
        \end{frame}
    
        \begin{frame}{ Парадокс на Олберт }
            \begin{columns}
                \begin{column}{0.5\textwidth}
                    \includegraphics[width=\textwidth]{images/big_bang_redshift.jpg}
                \end{column}
                \begin{column}{0.5\textwidth}
                    \begin{itemize}
                        \item Теория за Големия Взрив \begin{itemize}
                            \item Разширяващата се вселена
                            \item Червено отместване
                        \end{itemize}
                    \end{itemize}
                \end{column}
            \end{columns}
        \end{frame}
        
        \begin{frame}{ Проблеми на ОТО }
            \begin{columns}
                \begin{column}{0.5\textwidth}
                    \includegraphics[width=\textwidth]{images/big_bang_acc_expansion.jpg}
                \end{column}
                \begin{column}{0.5\textwidth}
                    \begin{itemize}
                        \item Теория за Големия Взрив \begin{itemize}
                            \item Разширяващата се вселена
                            \item Червено отместване
                            \item Тъмна енергия!!!
                        \end{itemize}
                    \end{itemize}
                \end{column}
            \end{columns}
        \end{frame}
        
        \begin{frame}{ Проблеми на ОТО }
            \begin{columns}
                \begin{column}{0.9\textwidth}
                    \includegraphics[width=\textwidth]{images/big_bang_dark_matter.png}
                \end{column}
            \end{columns}
        \end{frame}
        
        \begin{frame}{ Проблеми на ОТО }
            \begin{columns}
                \begin{column}{0.5\textwidth}
                    \includegraphics[width=\textwidth]{images/big_bang_dark_matter.png}
                \end{column}
                \begin{column}{0.5\textwidth}
                    \begin{itemize}
                        \item Тъмна енергия!!!
                        \item Тъмна материя!!!
                    \end{itemize}
                \end{column}
            \end{columns}
        \end{frame}
        
    \section{ Бъдеще }
        
        \begin{frame}{}
            \tableofcontents[currentsection]
        \end{frame}
        
        \begin{frame}{ Неутронни звезди }
                \begin{columns}
                    \begin{column}{0.8\textwidth}
                            \includegraphics[width=\textwidth]{images/neutron_star.png}
                    \end{column}
                \end{columns}
        \end{frame}
    
        \begin{frame}{ Неутронни звезди }
                \begin{columns}
                    \begin{column}{0.5\textwidth}
                        \includegraphics[width=\textwidth]{images/neutron_star.png}
                    \end{column}
                    \begin{column}{0.5\textwidth}
                        \begin{itemize}
                            \item $ 1.4 \div 3 M_{\odot} $, $ R \sim 10 $km
                            \item $ \sim 2000 $ наблюдавани
                        \end{itemize}
                    \end{column}
                \end{columns}
        \end{frame}
            
        \begin{frame}{ Скаларно-тензорни теории }
            \begin{block}{Общи параметри}
                \begin{itemize}
                    \item куплираща функция $ A(\varphi) = e^{1/2\beta \varphi}$
                    \item сила на куплиране $ a(\varphi) = \dfrac{\partial \ln A}{\partial \varphi} = \beta \varphi$
                    \item потенциал на полето $ V = U(\varphi) = 2m^2\varphi^2 + \lambda\varphi^4$
                \end{itemize}
            \end{block}
        \end{frame}
    
        \begin{frame}{ Спонтанна скаларизация, аналогия }
            
            \begin{columns}
                \begin{column}{0.9\textwidth}
                    \includegraphics[width=\textwidth]{images/spont_magn.png}
                \end{column}
            \end{columns}
            
        \end{frame}
        
        \begin{frame}{ СТТ, безмасова, безсамодействие }
            \begin{columns}
                \begin{column}{\textwidth}
                    \includegraphics[width=\textwidth]{images/STT_beta-6_m0_lambda0.png}
                \end{column}
            \end{columns}
        \end{frame}
    
        \begin{frame}{ СТТ, масова, безсамодействие }
            \begin{columns}
                \begin{column}{\textwidth}
                    \includegraphics[width=\textwidth]{images/STT_beta-6_m1e-2_lambda0.png}
                \end{column}
            \end{columns}
        \end{frame}
    
        \begin{frame}{ СТТ, безмасова, със самодействие }
            \begin{columns}
                \begin{column}{\textwidth}
                    \includegraphics[width=\textwidth]{images/STT_beta-6_m0_lambda1.png}
                \end{column}
            \end{columns}
        \end{frame}
        
	\begin{frame}
		\begin{center} 
            {\Huge Благодаря за вниманието ! } \\[5mm]
            {\small https://github.com/dpopchev/Open-Doors-Uni-2018 }
        \end{center}
	\end{frame}
	
\end{document}
